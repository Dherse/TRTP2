\documentclass[8pt]{article}

\usepackage[perpage]{footmisc}
\usepackage[utf8]{inputenc}
\usepackage{natbib}

\usepackage{natbib}
\usepackage[T1]{fontenc}

\usepackage{subfiles}
\usepackage{natbib}
\usepackage{gensymb}
\usepackage{graphicx}
\usepackage{float}
\usepackage{amsmath} 
\usepackage{array}
\usepackage{lineno}
\usepackage{amssymb}
\usepackage{wrapfig}
\usepackage{xcolor}
\usepackage[left=2.5cm,right=2.5cm,top=3cm,bottom=2.5cm]{geometry}
\usepackage{svg}
\usepackage{pdflscape}
\usepackage{enumitem}
\usepackage{listings}
\usepackage{color}
\usepackage{amsmath}
\usepackage{fancyhdr}
\usepackage{caption}
\usepackage{nicefrac}

\usepackage[scaled]{helvet}
\renewcommand\familydefault{\sfdefault} 
\usepackage{fontspec}

\pagestyle{fancy}
\fancyhead[L]{\includegraphics[scale=0.05]{assets/ucl.jpg}}
\fancyhead[R]{\includegraphics[scale=0.15]{assets/epl.jpg}}
\fancyfoot[L]{LINGI1341}
\fancyfoot[C]{Groupe 138}
\fancyfoot[R]{\thepage}
\definecolor{mygreen}{RGB}{28,172,0} % color values Red, Green, Blue
\definecolor{mylilas}{RGB}{170,55,241}

\renewcommand*{\thefootnote}{(\arabic{footnote})}


\usepackage{hyperref}
\hypersetup{
colorlinks=true,
linkcolor=black,
urlcolor=blue,
filecolor=green,
}

\newcommand*\mean[1]{\overline{#1}}
\newcommand*{\thead}[1]{\multicolumn{1}{c}{\bfseries #1}}

\setlength{\parindent}{1em}
\setlength{\parskip}{0.2em}

\begin{document}

\subfile{sections/cover}
\newpage

\section{Abstract}

Nous avons conçus une implémentation du receveur TRTP multithreadée capable de saturer une connexion 1 GbE et d'utiliser en grande partie une connexion 10 GbE.
Dans nos tests, la vitesse de transfer maximale observée était de 700 $\nicefrac{MiB}{s}$\footnote{\hyperref[sec:performance]{cfr. section sur les performances}}. Et celle-ci
était limitée non pas par notre implémentation mais notre capacité à générer les paquets du côté des \textit{senders} assez rapidement. Ces performances sont possibles grâce à l'usage de
\textit{\hyperref[sec:syscalls]{syscalls avancés}}, d'opération \textit{\hyperref[sec:atomics]{atomiques}} et l'usage de multiple \textit{\hyperref[sec:pipelines]{pipelines}} traitant
l'information en parallèle tout en restant \textit{thread safe} et libre de toute fuite mémoire. De plus notre implémentation est accompagnée d'une suite
étendue de tests couvrant la grande majorité du code et d'une documentation complète détaillant l'architecture et le fonctionnement des différents composants
utilisés.

\newpage

\tableofcontents
\newpage

\subfile{sections/introduction}
\newpage

\subfile{sections/architecture}
\newpage

\subfile{sections/performances}
\newpage

\bibliographystyle{unsrt}
\bibliography{project}


\end{document}