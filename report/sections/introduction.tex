\documentclass[../main.tex]{subfiles}

\begin{document}

\section{Introduction}

The goal of this project was to implement a receiver based on the TRTP protocol capable of handling multiple clients at the same time. 
To try and have a more interesting and more in depth analysis we decided to realise an optionally multithreaded version. This means it 
can use multiple threads if desired or run on a single one. In the first mode we can reach with limited software/hardware tinkering speeds 
exceeding 2$\nicefrac{Gb}{s}$ and in single threaded mode speeds exceeding 1$\nicefrac{Gb}{s}$ easily saturating a typical 1GbE card. In 
addition, our implementation can treat hundreds of concurrent connections with an excellent success rate of over 99\%\footnote{Note that 
all of the failures are due to sender unable to cope with the data rate}. This means our implementation fulfills the requirements and 
more by providing a high performance and reliable implementation.

In this document the architectural decisions and the techniques employed will be explained followed by in-depth performance analysis 
in different conditions including stress tests and link simulation. The end goal is to understand why a multithreaded implementation 
is interesting and why the specifics of our implementation give great performance with reasonable CPU usage. It will also try to explain
 the hard limit seen at 2.1$\nicefrac{Gb}{s}$ during testing.

\end{document}