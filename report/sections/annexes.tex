\documentclass[../main.tex]{subfiles}

\begin{document}

\section{Annexes}
\label{sec:annexes}

\subsection{Tests d'interopérabilité}
\label{sec:interop}

Nous avons effectué des tests d'interopérabilité avec deux groupe : le groupe numéro 85 et le groupe numéro 93. 

Notre premier test fut celui avec le groupe 93. Lors de celui-ci, nous sommes parvenus à transférer des petits fichiers, et ce peu 
importe les arguments donnés au receiver (sequencial, un seul handler, etc.) sans aucun problème. Néanmoins, lors de plus gros transferts, 
des problèmes sont apparus.
Tout d'abord, il arrivait que notre receiver \textit{segfault}, ce qui, par ailleurs, entraînait un \textit{segfault} chez le \textit{sender}.
Une fois ce \textit{segfault} trouvé et corrigé, un \textit{livelock} avait l'air de se mettre en place de manière erratique, sauf lorsque 
notre receiver fonctionnait en mode \textit{sequencial}. Après analyse de la discussion \textit{sender} - \textit{receiver}, nous avons conclu
que ce comportement était le résultat du réordonnancement des paquets dû au multithreading du \textit{receiver} combiné avec la façon dont le 
\textit{sender} retransmettait sa \textit{window} en fonction du \textit{timestamp} des ACKs. Néanmoins, puisque le réordonnancement de paquets 
est quelque chose à quoi le \textit{sender} est sencé s'attendre, le problème était du côté du \textit{sender}.

Notre deuxième test s'est passé de manière beaucoup plus fluide puisque le groupe 85 avait comme nous déjà corrigé les bugs d'interopérabilité.
Nous avons donc pus transmettre nos fichiers sans aucun problèmes.


\end{document}
