\documentclass[../main.tex]{subfiles}

\begin{document}

\section{Architecture}
\label{sec:architecture}

Dès le début du projet, le but fixé était de réaliser une implémentation multithreadée afin de maximiser les performances. Bien que l'architecture 
 originellement conçue était très naïve, elle s'est avérée être un bon point de départ pour ce qu'est notre architecture finale. Celle-ci est flexible
fonctionnant autant en mode \hyperref[sec:sequential]{\textit{séquentiel}}, c'est à dire sur un unique thread, que sur beaucoup de thread séparés. De plus, le fonctionnement parallèle
est amélioré par la présence de deux fichiers de configurations permettant d'optimiser d'avantage l'exécution sans devoir changer le code source.

L'architecture se base sur la possibilité en UDP d'écrire et de lire depuis un même socket depuis plusieurs thread en même temps. Ceci est possible car l'UDP
n'est pas un protocole qui garantit l'ordre d'arrivée des paquets. Notre programme utilise donc une série de threads pour lire depuis le socket et une série
de thread pour traiter les données et envoyer les paquets de (N)ACK. Ce premier type sera appellé \textit{receiver} tandis que le deuxième sera appellé \textit{handler}.

Chacun de ces threads a ensuite été amélioré par l'usage de \textit{sycalls} avancés permettant de drastiquement diminuer non seulement le nombre de \textit{syscalls}
exécutés mais également le nombre de paquets de types ACK que le receveur doit émettre. Ceci sera expliqué dans plus de détails dans les sections
qui suivent.

Ces threads communiquent entre eux à l'aide de deux queues de communication. La première sert à envoyer les paquets depuis le \textit{receiver} au \textit{handler}
et la deuxième fait le chemin inverse. Bien que cela soit plus complexe, cela permet une meilleure utilisation de la mémoire allouée en réutilisant les
\textit{buffers} précédemment alloués. De plus ce \textit{stream} utilise des opérations atomiques\footnote{ Opérations \textit{thread safe} ne requérant pas de verrou}
lors de l'ajout d'éléments afin d'augmenter les performances et de diminuer la latence dans le \textit{receiver}.

\subsection{Table de hachage type \textit{linear probing}}
\label{sec:hash_table}

Le but de cette table de hachage ou \textit{hash table} est de stocker la liste des clients concourants et de pouvoir rapidement identifier les nouvelles
connexions. Elle stocke une structure \textit{client\_t} qui contient les données relatives à une connexion telle que l'adresse IP, le port et l'état
de la fenêtre de réception. Cette table est partagée entre tous les \textit{receivers} et est correctement protégée par un \textit{mutex}.

La table de hachage donne un énorme avantage par rapport aux autres implémentations car elle est, en moyenne, en complexité temporelle $\mathcal{O}(1)$\cite{linear_probing}
sans requérir l'usage d'un \textit{socket} par connexion qui pourrait causer problème pour un très grand nombre de connexion concourantes\footnote{
    Le nombre de \textit{file descriptor} maximum autorisé par le système d'exploitation pourrait être atteint.}.

Afin de simplifier l'implémentation de celle-ci, notre table de hachage n'utilise pas l'adresse IP du client mais uniquement le port de celui-ci comme
clé. Cela rend le calcul de l'index très simple : $Port \% M$ où $M$ est la capacité de la table de hachage. Ensuite, lors du \textit{linear probing}, %TODO : FAUX !!! c'est M-1 parce que l'index commence à 0
l'adresse IP est comparée en plus du port afin de garantir l'unicité de chaque client. De plus cette comparaison est expressément faite avec
la fonction \textit{memcmp} afin d'en augmenter les performances\footnote{ \textit{memcmp} utilise SSE 4.2 comme observé dans le \textit{\hyperref[sec:annexes_call_graph]{call graph}}.}.
Ce qui veut donc dire qu'un maximum théorique de $65536$ connexions concourantes sans collisions peuvent être traité. En pratique ce nombre
est plus petit. De plus, avec une implémentation utilisant un \textit{socket} par client, la limite typique sur une machine linux serait
d'environ $511$\footnote{ $\frac{1}{2} \cdot (ulimit_{typique} - 1)$ où $ulimit_{typique}$ est la valeur typique du nombre maximal de \textit{file descriptor} par processur,
 le $\frac{1}{2}$ est dû au socket à l'ouverture de fichier et le $1$ fait référence au socket global. } connexions. Alors
que notre implémentation pourrait en traiter $1024 - N$\footnote{ $ulimit_{typique} - N$ où $N$ est le nombre de \textit{pipelines}.} soit plus du double pour
un nombre résonnable de \textit{pipelines}.

\newpage

\subsection{\textit{Streams} - queue de communication} %%TODO : queue en francais c'est moche => file ? channel ?, autre ?
\label{sec:steams}

Comme mentionné ci-dessus, la communication entre les différents \textit{threads} est assurée par des queues de communications appelés \textit{streams}.
Il s'agit de liste chaînées d'éléments de types \textit{s\_node\_t}. Elles utilisent des opérations atomiques pour l'insertion ce qui permet
à l'opération \textit{enqueue} de ne bloquer que très rarement en faisant du \textit{busy waiting} au lieu d'employer des \textit{mutexes}. Cela permet %%TODO : mutexes est peut etre le pluriel en Anglais mais en francais c'est moche
au \textit{receiver} de ne jamais attendre pour insérer dans le stream. En revanche l'opération \textit{dequeue} est protégée par un \textit{mutex} et
emploie une variable de condition\footnote{ Permet de déverouiller un \textit{mutex} en attendant d'être notifié. Un autre \textit{thread} peut ensuite notifier %%TODO : conditionnal variable = variable conditionnelle; ta traduction là c'est moche
la condition et le \textit{mutex} sera automatiquement verouillé. Il est intéressant de noter que la documentation mentionne qu'il est autorisé %%TODO : tu as pas inversé déverrouiller et verrouillé ?
de notifier une variable conditionelle sans posséder le \textit{mutex} ce qui permet de maintenir le \textit{dequeue} sans verrou. } pour attendre la dispobilité de nouveaux éléments.

\subsection{\textit{Syscalls} avancés - \textit{recvmmsg} \& \textit{sendmmsg}}
\label{sec:syscalls}

Les operations typiques effectuées sur des \textit{sockets} sont \textit{recv}, \textit{recvfrom}, \textit{send} et \textit{sendto}. Bien qu'il n'y ait rien
de mal avec ces implémentations, l'usage des \textit{syscalls} \textit{sendmmsg} et \textit{recvmmsg} sont suggerés comme amélioration disponible\cite{that_awesome_paper} %% TODO : disponible fait de trop 
de ces fonctions de base. Dans notre implémentation, l'usage de \textit{recvmmsg} est particulièrement intéressant car il permet de recevoir plus d'un paquet à la fois
rendant possible la réception d'une fenètre complète en un \textit{syscalls} au lieu de $31$ avec \textit{recvfrom}/\textit{recv}. De plus, \textit{recvmmsg} permet
de lire l'adresse IP et le port du client afin d'effectuer une recherche dans la table de hachage. Et \textit{sendmmsg} permet d'envoyer un seul \textit{ACK} pour
une série de paquets reçus mais également pour chaque paquet hors de séquence ou corrompu. Ce qui diminue à nouveau le nombre de \textit{syscalls} effectués.

Il est intéressant de noter que cet avantage est très important\footnote{ cfr. partie performance} pour un nombre limité de \textit{receiver} mais diminue lorsque  %%TODO : expliquer qu'avec une augmentation de clients , les collisions augmentent et donc le nombre d'ACK aussi
ce nombre augmente. De même, lorsque le nombre de client augmente, l'efficacité de ce système diminue pour faire l'aquisition d'une fenètre complète. 
Cependant, l'avantage premier qui est de diminuer le nombre de \textit{syscalls} reste constant. C'est donc purement l'avantage du point de vue de
l'usage de la bande passante qui diminue et non celui du nombre réduit de \textit{syscalls}. %%TODO : cette phrase est très bizarrement construite.

\subsection{\textit{Buffers} - implémentation de la window}
\label{sec:buffer}

Cette structure consiste d'un simple tableau acompagné de quelques informations comme l'état d'utilisation d'un emplacement. Il y a une et une seule instance d'un \textit{buffer} par client.
Il est facile de repérer que la fenètre de réception ayant une taille maximale de $31$ elle peut elle même être vue comme une ``table de hachage'' dans le
sens ou il est possible de prendre les 5 \textit{LSBs}\footnote{ Dans \textit{tests/buffer\_test.c} se trouve un test testant toutes les fenètres possibles.} 
du \textit{seqnum} pour obtenir l'index d'un \textit{sequence number} dans un array de taille $32$. Il suffit ensuite de limiter le nombre maximal d'éléments dans la table 
à $31$ et la fenêtre de réception est implémentée. L'avantage de cette implémentation est sa vitesse car elle est ordonnée et en temps constant, tant à l'insertion
qu'à la recherche d'un élément, et les conflits sont impossibles. De plus, il suffit de regarder si un espace est occupé pour voir si le paquet reçu est un
doublon ou pas. Cependant, le \textit{buffer} ne vérifie pas si un nombre fait partie de la fenêtre actuelle, cela veut dire que le numéro de séquence doit
être vérifié à l'avance, ceci est fait en utilisant une \textit{LUT} (\textit{LookUp Table}), elle a l'avantage de ne pas avoir de logique la rendant simple
à comprendre et un temps de recherche constant sans branchement ou opération mathématique.

\subsection{\textit{Receiver thread}}
\label{sec:receiver}

Le \textit{receiver thread} est responsable de la réception des paquets, l'identification de l'émetteur du paquet au travers de la table de hachage
et de les transmettre au travers du \textit{stream} aux \textit{handlers}. Il s'occupe également d'identifier les nouveaux clients et 
des les ajouter à la table ou de les rejetter si celle-ci est pleine.

De plus, le receiver fait un groupement basique des paquets par client afin d'utiliser un acquitement pour plusieurs paquets. Ceci est fait
en itérant sur la liste de paquet retourné par \textit{recvmmsg}, tant que le client est le même, les paquets sont groupés, dés qu'un paquet
appartient à un autre client, un nouveau \textit{s\_node\_t} est utilisé et le précédent est ajouté au \textit{stream}.

\subsection{\textit{Handler thread} - génération des acquitements}
\label{sec:handler}

Le \textit{handler thread}, quand à lui, est responsable de prendre un paquet brut du réseau et son émetteur et de le traiter en vérifiant tout %%TODO : "le traiter" fait référence à émetteur au lieu de paquet
d'abord sa validité, c'est-à-dire s'il rentre dans la fenêtre de réception, si ses \textit{CRC32} sont valides, etc. Ensuite, si le paquet
est invalide, un ACK est généré pour être envoyé au \textit{sender}. En revanche, si le paquet est valide, il est ajouté au \textit{buffer} %% TODO : expliquer qu'envoyer un ACK en cas de paquet non valide n'est pas prévu par le protocole mais permets d'éviter des deadlocks dans la connection
du client. Une fois que tous les paquets dans le \textit{s\_node\_t} ont été traités, le \textit{handler} itère si possible sur la fenêtre en partant %%TODO : manque de consistance : tu utilisait le terme "window" mais maintenant tu mets fenêtre. c'est un peu bizarre
du \textit{sequence number} le plus bas jusqu'à ce que $31$ éléments aient été traités ou jusqu'au premier élémént vide. Ces éléments sont ensuite
écrits dans le fichier de sortie du client en passant par un \textit{buffer} intermédiaire afin de ne faire qu'un seul appel à \textit{fwrite}
par groupe de paquets écrits. Si plus d'un paquet peut être écrits, un ACK est généré pour cette liste de paquets en utilisant le \textit{seqnum}
le plus haut de la fenètre. Enfin, l'index le plus bas de la fenêtre est mis à jour.


Si le dernier paquet se trouvant dans le \textit{buffer} a une longueur de zéro, le client est marqué pour suppresion et son fichier de sortie
fermé. Tout paquet arrivant pour ce client tant qu'il n'aura pas été supprimé causera l'envoi d'un ACK notifiant que le paquet de fin de transmission
a correctement été reçu. Le client n'est pas immédiatement détruit afin de permettre au \textit{sender} de réenvoyer ce paquet si le
paquet d'acquitement avait été perdu.

\subsection{Multiple \textit{Pipelines} \& \textit{Affinities} - optimisation de l'exécution}
\label{sec:pipelines}

Lors de tests de performances une étrange limite aux alentours de 370 $\nicefrac{MiB}{s}$ a rapidement été découverte. Ayant déjà atteint
le but original de 100 $kpps$ sans trop de difficulté, une amélioration jusqu'à 1 $Mpps$ semblait raisonable, plusieurs sources\cite{that_awesome_paper,1mmps_article}
suggéraient d'utiliser plusieurs \textit{socket} en parallèle en utilisant les paramètres \textit{SO\_REUSEADDR} et \textit{SO\_REUSEPORT} afin d'éviter
les conflits entre les sockets. En plus d'augmenter le nombre de \textit{sockets}, un fichier de configuration totalement optionnel a également été créé
permettant de définir des groupes de un ou plusieurs \textit{receivers} avec un ou plusieur \textit{handlers} partageant un \textit{stream} et un \textit{socket}.
Ceci s'est avéré être une idée judicieuse car cela a permis de largment dépasser les 1,5 $Mpps$.

Ce premier fichier de configuration est accompagné d'un deuxième fichier de configuration, lui aussi tout à fait optionnel, qui permet de définir les affinités
de chaque \textit{receiver} et de chaque \textit{handler}. Cela revient à manuellement définir le coeur physique du \textit{CPU} sur lequel le \textit{thread}
doit tourner. Cela permet de faire deux choses très importantes pour maximiser les performances, la première, cela permet de stabiliser la vitesse de transfer
en évitant que le \textit{scheduler} déplace les \textit{threads} pendant l'exécution mais cela permet aussi de bénéficier de la \textit{core locality}.

Le nombre de \textit{receiver} et \textit{handler threads} est contrôllé par deux paramètres additionels respectivements \textit{N} et \textit{n}. Le nombre
de chaque référencés dans les fichiers de configurations doit être le même que les paramètres définis par ces options. Les valeurs de bases de ces options sont
1 et 2 afin de maintenir un rapport (trouvé expérimentalement) de 0.5 qui semble être idéal dans la plupart des cas.

\subsubsection{\textit{Core locality}}

Dans les processeurs modernes, les coeurs physiques sont souvant groupé par groupe de 4 ou plus. Ce groupe de coeurs partage une même cache, controlleur mémoire, etc.
Cela veut dire que la communication entre ces coeurs se fera plus rapidement qu'entre des coeurs de groupes séparés. Ce phénomène est normalement gênant, mais
en utilisant l'affinité et la configuration des \textit{streams}, il est possible d'agencer les \textit{threads} d'une façon optimale sans procéder à des essais
mais juste en comprenant l'architecture du processeurs. A nouveau, c'est une optimisation tout à fait optionelle qui est simplement utilisée afin de maximiser
les performances dans les tests.

\subsection{Mode séquentiel}
\label{sec:sequential}

Le mode séquentiel est activé en utilisant une l'option \textit{s} qui a été rajoutée au \textit{receiver}. Il permet simplement de faire 
tourner notre implémentation sans le moindre \textit{thread} (en dehors de la \textit{main}). Ce mode a toujours de bonnes performances et ignore totalement 
les fichiers de configurations et paramètres additionels. Il a été ajouté afin de respecter totalement la spécification et le conseil de ne pas utiliser de 
threads. Cependant, ce n'est pas le mode de base.

\end{document}